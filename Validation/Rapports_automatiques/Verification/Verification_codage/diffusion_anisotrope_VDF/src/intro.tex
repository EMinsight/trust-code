This document presents a validation regarding the implementation of the anisotropic diffusion operator in the TRUST platform. Only the VDF discretisation is considered in the present work. Different simulations are performed and numerical results at steady-state are compared to analytical solutions for validations. Two test cases are investigated in this work. The first one is to validate a 2D heat conduction problem with an anisotropic discontinuous conductivity tensor, while the second with an anisotropic non-uniform conductivity tensor. In all cases, the conductivity tensor is diagonal as far as the code accepts currently a tensor with eigenvectors aligned with the VDF mesh.  For the second test case, the sensitivity of the mesh and the influence of the anisotropy ratio is also reported. The reader is kindly invited to check PH. Mairea and J. Breil for a detailed discussion concerning the studied test cases, \textit{https://hal.archives-ouvertes.fr/hal-00605548/document}).

The 2D heat conduction equation is expressed as
\begin{equation}\label{cond}
\rho c_p \frac{\partial T}{\partial t} - \nabla\cdot (\kappa\nabla T)=\dot{q_v}.
\end{equation}
$\rho$ [kg.m$^{-3}$] denotes the density field, $c_p$ [J.kg$^{-1}$.K$^{-1}$] the specific heat capacity, $\kappa$ [W.K$^{-1}$.m$^{-1}$] the heat conductivity tensor and $\dot{q_v}$ [W.m$^{-3}$] the volumetric heat source term. The domain considerd is $\Omega=[0,1]^2 \in \mathbb{R}^2$. Both $\rho$ and $c_p$ fields are assumed constant and uniform; their values are set to unity. 

Regarding the source term $\dot{q_v}$ and the conductivity tensor, the employed formulations are dependant of the treated test case and will be discussed later in each section. The SYMPY library from python3 is used to define the exact source term based on the known analytical solution (see the validation file diffusion\_anisotrope\_VEF for further information).

