This document constitutes the generic guide for \textbf{TRUST software} and his \textbf{Baltik projects}.\\

\trust is a thermohydraulic calculation modular software package for CFD simulations of incompressible monophasic/diphasic flow.\\

You could create new project based on \trust plateform. Theses projects were named \textbf{"BALTIK"} projects.\\

The two currently available modules include a VDF calculation module "Finite Difference Volume" and a VEF calculation module "Finite Element Volume". \\

The VDF and VEF validated modules are designed to process the 2D or 3D flow of Newtonian, incompressible, weakly expandable fluids the density of which is a function of a local temperature and concentration values (Boussinesq approximation).



%%%%%%%%%%%%%%%%%%%%%%%%%%%%%%%%%%%%%%%%%%%%%%%%%%%%%%%%%%%%%%%
\section{Before TRUST: a brick software named Trio\_U}
%%%%%%%%%%%%%%%%%%%%%%%%%%%%%%%%%%%%%%%%%%%%%%%%%%%%%%%%%%%%%%%

\trust was born from the cutting in two pieces of \textbf{Trio\_U} software.
\textbf{Trio\_U} was a software brick based on the \textbf{Kernel} brick (which contains the equations, space discretizations, numerical schemes, parallelism...) and used by other CEA applications (cf Figure \ref{TrioU}).

\begin{figure}[h!]
\begin{center}
\begin{tikzpicture}[scale=2, line width=1pt]
% box Kernel
\coordinate (A) at (0,0) ;
\coordinate (B) at (6.25,0) ;
\coordinate (C) at (6.25,0.5) ;
\coordinate (D) at (0,0.5) ;
\coordinate (E) at (0.25,0.75) ;
\coordinate (F) at (6.5,0.75) ;
\coordinate (G) at (6.5,0.25) ;
\draw[black,fill=orange!80] (A) -- (B) -- (C) -- (D) -- cycle ;
\draw[black,fill=orange!80] (D) -- (C) -- (F) -- (E) -- cycle ;
\draw[black,fill=orange!80] (B) -- (C) -- (F) -- (G) -- cycle ;
\draw (3,0.1) node[above]{\textbf{Kernel}} ;
%% box Trio_U
\begin{scope}
\coordinate (A1) at (0,0.625) ;
\coordinate (B1) at (3.125,0.625) ;
\coordinate (C1) at (3.125,1.125) ;
\coordinate (D1) at (0,1.125) ;
\coordinate (E1) at (0.25,1.375) ;
\coordinate (F1) at (3.375,1.375) ;
\coordinate (G1) at (3.375,0.875) ;
\draw[black,fill=orange!80] (A1) -- (B1) -- (C1) -- (D1) -- cycle ;
\draw[black,fill=orange!80] (D1) -- (C1) -- (F1) -- (E1) -- cycle ;
\draw[black,fill=orange!80] (B1) -- (C1) -- (F1) -- (G1) -- cycle ;
\draw (1.5,0.7) node[above]{\textbf{Trio\_U}} ;
\end{scope}
% box MPCube
\begin{scope}[xshift=3.25 cm]
\coordinate (A1) at (0,0.625) ;
\coordinate (B1) at (1.5,0.625) ;
\coordinate (C1) at (1.5,1.125) ;
\coordinate (D1) at (0,1.125) ;
\coordinate (E1) at (0.25,1.375) ;
\coordinate (F1) at (1.75,1.375) ;
\coordinate (G1) at (1.75,0.875) ;
\draw[black,fill=mauve!80] (A1) -- (B1) -- (C1) -- (D1) -- cycle ;
\draw[black,fill=mauve!80] (D1) -- (C1) -- (F1) -- (E1) -- cycle ;
\draw[black,fill=mauve!80] (B1) -- (C1) -- (F1) -- (G1) -- cycle ;
\draw (0.75,0.75) node[above]{\textbf{MPCube}} ;
\end{scope}
%% box MC2
\begin{scope}[xshift=0 cm, yshift=0.625 cm]
\coordinate (A1) at (0,0.625) ;
\coordinate (B1) at (1.5,0.625) ;
\coordinate (C1) at (1.5,1.125) ;
\coordinate (D1) at (0,1.125) ;
\coordinate (E1) at (0.25,1.375) ;
\coordinate (F1) at (1.75,1.375) ;
\coordinate (G1) at (1.75,0.875) ;
\draw[black,fill=red!80] (A1) -- (B1) -- (C1) -- (D1) -- cycle ;
\draw[black,fill=red!80] (D1) -- (C1) -- (F1) -- (E1) -- cycle ;
\draw[black,fill=red!80] (B1) -- (C1) -- (F1) -- (G1) -- cycle ;
\draw (0.75,0.75) node[above]{\textbf{MC2}} ;
\end{scope}
% box Genepi+
\begin{scope}[xshift=1.625 cm, yshift=0.625 cm]
\coordinate (A1) at (0,0.625) ;
\coordinate (B1) at (1.5,0.625) ;
\coordinate (C1) at (1.5,1.125) ;
\coordinate (D1) at (0,1.125) ;
\coordinate (E1) at (0.25,1.375) ;
\coordinate (F1) at (1.75,1.375) ;
\coordinate (G1) at (1.75,0.875) ;
\draw[black,fill=vert] (A1) -- (B1) -- (C1) -- (D1) -- cycle ;
\draw[black,fill=vert] (D1) -- (C1) -- (F1) -- (E1) -- cycle ;
\draw[black,fill=vert] (B1) -- (C1) -- (F1) -- (G1) -- cycle ;
\draw (0.75,0.7) node[above]{\textbf{GENEPI+}} ;
\end{scope}
\end{tikzpicture}
\caption{Trio\_U: brick software}
\label{TrioU}
\end{center}
\end{figure}

We could create new project based on Kernel brick or \textbf{Trio\_U} brick. 
Theses projects were named \textbf{"BALTIK"} projects: "\textbf{B}uild an \textbf{A}pplication \textbf{L}inked to \textbf{T}r\textbf{I}o\_U \textbf{K}ernel".\\

In 2015, \textbf{Trio\_U} was divided in two parts: \trust and \textbf{TrioCFD}.
\begin{itemize}
\item \trust is a new platform, its name means: "\textbf{TR}io\_\textbf{U} \textbf{S}oftware for \textbf{T}hermohydraulics",
\item \textbf{TrioCFD} is a BALTIK project based on \trust, which contains the following models: FT, Radiation, LES, zoom...
\end{itemize}

Here is the structure of \trust platform (cf Figure \ref{TRUST}):

\begin{figure}[h!]
\begin{center}
\begin{tikzpicture}[scale=2, line width=1pt]
% box TRUST
\coordinate (A) at (0,0) ;
\coordinate (B) at (7,0) ;
\coordinate (C) at (7,0.5) ;
\coordinate (D) at (0,0.5) ;
\coordinate (E) at (0.25,0.75) ;
\coordinate (F) at (7.25,0.75) ;
\coordinate (G) at (7.25,0.25) ;
\draw[black,fill=orange!80] (A) -- (B) -- (C) -- (D) -- cycle ;
\draw[black,fill=orange!80] (D) -- (C) -- (F) -- (E) -- cycle ;
\draw[black,fill=orange!80] (B) -- (C) -- (F) -- (G) -- cycle ;
\draw (3.5,0) node[above]{\trust platform: Kernel/Code Coupling Interface / V \& V Tools / doc} ;
%% box TrioCFD
\begin{scope}
\coordinate (A1) at (0,0.625) ;
\coordinate (B1) at (1.5,0.625) ;
\coordinate (C1) at (1.5,1.125) ;
\coordinate (D1) at (0,1.125) ;
\coordinate (E1) at (0.25,1.375) ;
\coordinate (F1) at (1.75,1.375) ;
\coordinate (G1) at (1.75,0.875) ;
\draw[black,fill=DeepSkyBlue!80] (A1) -- (B1) -- (C1) -- (D1) -- cycle ;
\draw[black,fill=DeepSkyBlue!80] (D1) -- (C1) -- (F1) -- (E1) -- cycle ;
\draw[black,fill=DeepSkyBlue!80] (B1) -- (C1) -- (F1) -- (G1) -- cycle ;
\draw (0.75,0.75) node[above]{\textbf{TrioCFD}} ;
\end{scope}
% box TrioMC2
\begin{scope}[xshift=1.625 cm]
\coordinate (A1) at (0,0.625) ;
\coordinate (B1) at (1.5,0.625) ;
\coordinate (C1) at (1.5,1.125) ;
\coordinate (D1) at (0,1.125) ;
\coordinate (E1) at (0.25,1.375) ;
\coordinate (F1) at (1.75,1.375) ;
\coordinate (G1) at (1.75,0.875) ;
\draw[black,fill=red!80] (A1) -- (B1) -- (C1) -- (D1) -- cycle ;
\draw[black,fill=red!80] (D1) -- (C1) -- (F1) -- (E1) -- cycle ;
\draw[black,fill=red!80] (B1) -- (C1) -- (F1) -- (G1) -- cycle ;
\draw (0.75,0.75) node[above]{\textbf{TrioMC2}} ;
\end{scope}
% box Genepi+
\begin{scope}[xshift=3.248 cm]
\coordinate (A1) at (0,0.625) ;
\coordinate (B1) at (1.5,0.625) ;
\coordinate (C1) at (1.5,1.125) ;
\coordinate (D1) at (0,1.125) ;
\coordinate (E1) at (0.25,1.375) ;
\coordinate (F1) at (1.75,1.375) ;
\coordinate (G1) at (1.75,0.875) ;
\draw[black,fill=vert] (A1) -- (B1) -- (C1) -- (D1) -- cycle ;
\draw[black,fill=vert] (D1) -- (C1) -- (F1) -- (E1) -- cycle ;
\draw[black,fill=vert] (B1) -- (C1) -- (F1) -- (G1) -- cycle ;
\draw (0.75,0.7) node[above]{\textbf{Genepi+}} ;
\end{scope}
%% box MPCube
\begin{scope}[xshift=4.875 cm]
\coordinate (A1) at (0,0.625) ;
\coordinate (B1) at (1.5,0.625) ;
\coordinate (C1) at (1.5,1.125) ;
\coordinate (D1) at (0,1.125) ;
\coordinate (E1) at (0.25,1.375) ;
\coordinate (F1) at (1.75,1.375) ;
\coordinate (G1) at (1.75,0.875) ;
\draw[black,fill=mauve!80] (A1) -- (B1) -- (C1) -- (D1) -- cycle ;
\draw[black,fill=mauve!80] (D1) -- (C1) -- (F1) -- (E1) -- cycle ;
\draw[black,fill=mauve!80] (B1) -- (C1) -- (F1) -- (G1) -- cycle ;
\draw (0.75,0.75) node[above]{\textbf{MPCube}} ;
\end{scope}
\end{tikzpicture}
\caption{TRUST platform \& his BALTIKs}
\label{TRUST}
\end{center}
\end{figure}

\Note that: \textbf{Trio\_U = TRUST + TrioCFD}.
%%%%%%%%%%%%%%%%%%%%%%%%%%%%%%%%%%%%%%%%%%%%%%%%%%%%%%%%%%%%%%%



%%%%%%%%%%%%%%%%%%%%%%%%%%%%%%%%%%%%%%%%%%%%%%%%%%%%%%%%%%%%%%%
\section{Software little history}
%%%%%%%%%%%%%%%%%%%%%%%%%%%%%%%%%%%%%%%%%%%%%%%%%%%%%%%%%%%%%%%

\trust is developed at the CEA/DEN/DANS/DM2S/STMF service.
The project starts in 1994 and improved versions were built since:
\begin{itemize}
\item 1994: start of the project Trio\_U
\item 01/1997: v1.0 (VDF only)
\item 06/1998: v1.1 (VEF version)
\item 04/2000: v1.2 (parallel version)
\item 07/2001: v1.3 (radiation model)
\item 11/2002: v1.4 (new LES turbulence models)
\item 02/2006: v1.5 (VDF/VEF Front Tracking)
\item 10/2009: v1.6 (data structure revamped)
\item 06/2015: v1.7 (cut into \trust and \textbf{TrioCFD} + switch to open source)
\end{itemize}
%%%%%%%%%%%%%%%%%%%%%%%%%%%%%%%%%%%%%%%%%%%%%%%%%%%%%%%%%%%%%%%




%%%%%%%%%%%%%%%%%%%%%%%%%%%%%%%%%%%%%%%%%%%%%%%%%%%%%%%%%%%%%%%
\section{Data file}
%%%%%%%%%%%%%%%%%%%%%%%%%%%%%%%%%%%%%%%%%%%%%%%%%%%%%%%%%%%%%%%

To launch a calculation with \trust, you need to write a "data file" which is an input file for \trust and will contain all the informations about your simulation.
Data files are written following some rules as shown above. But their language is not a programming language, users can't make loops or switch...\\

\Note that:
\begin{itemize}
\item lignes between \textcolor{blue}{\# ... \#} and \textcolor{blue}{/* ... */} are comments,
\item in that document, words in \textbf{bold} are \trust keywords, you can enlight them in your file editor with the command line (details in section \ref{Run}):\\
\texttt{> trust -config nedit|vim|emacs}
\item braces "\{ \}" are elements that \trust reads and interpretes, so don't forget them and \underline{put space before and} \underline{after them},
\item elements between bracket "[ ]" are optionnal.
\end{itemize}

%%%%%%%%%%%%%%%%%%%%%%%%%%%%%%%%%%%%%%%%%%%%%%%%%%%%
\subsection{Data file example: base blocks} \label{data}
%%%%%%%%%%%%%%%%%%%%%%%%%%%%%%%%%%%%%%%%%%%%%%%%%%%%
Here is a frame of basic sequential data file:\\

\fbox{ \begin{minipage}[c]{1\textwidth}
\begin{alltt}
\textcolor{blue}{\# Dimension 2D or 3D \#}

{\bf{Dimension}} 2
\end{alltt}
\end{minipage}}
\vspace{0.1cm}



\fbox{ \begin{minipage}[c]{1\textwidth}
\begin{alltt}
\textcolor{blue}{\# Problem definition \#}

{\bf{Pb\_hydraulique}} \textit{my\_problem}
\end{alltt}
\end{minipage}}
\vspace{0.1cm}



\fbox{ \begin{minipage}[c]{1\textwidth}
\begin{alltt}
\textcolor{blue}{\# Domain definition \#}

{\bf{Domaine}} \textit{my\_domain}
\end{alltt}
\end{minipage}}
\vspace{0.1cm}



\fbox{ \begin{minipage}[c]{1\textwidth}
\begin{alltt}
\textcolor{blue}{\# Mesh \#}

{\bf{Read\_file}} \textit{my\_mesh.geo} ;
\end{alltt}
\end{minipage}}
\vspace{0.1cm}



\fbox{ \begin{minipage}[c]{1\textwidth}
\begin{alltt} 
\textcolor{blue}{\# For parallel calculation only! \#}

\textcolor{blue}{\# For the first run: partitioning step \#}

\textcolor{blue}{{\bf{Partition}} \textit{my\_domain}}

\textcolor{blue}{\{}

\textcolor{blue}{\hspace{1cm}    {\bf{Partition\_tool}} \textit{partitioner\_name} \{ \textit{option1 option2 ...} \}}

\textcolor{blue}{\hspace{1cm}    {\bf{Larg\_joint}} \textit{2}}

\textcolor{blue}{\hspace{1cm}    {\bf{zones\_name}} \textit{DOM}}

\textcolor{blue}{\hspace{1cm}    ...}

\textcolor{blue}{\}}

\textcolor{blue}{{\bf{End}}}
\end{alltt}
\end{minipage}}
\vspace{0.1cm}



\fbox{ \begin{minipage}[c]{1\textwidth}
\begin{alltt} 
\textcolor{blue}{\# For parallel calculation only! \#}

\textcolor{blue}{\# For the second run: read of the sub-domains \#}

\textcolor{blue}{\# {\bf{Scatter}} \textit{DOM}{\bf{.Zones}} \textit{my\_domain} \#}
\end{alltt}
\end{minipage}}
\vspace{0.1cm}



\fbox{ \begin{minipage}[c]{1\textwidth}
\begin{alltt}
\textcolor{blue}{\# Discretization on hexa or tetra mesh \#}

{\bf{VDF}} \textit{my\_discretization}
\end{alltt}
\end{minipage}}
\vspace{0.1cm}



\fbox{ \begin{minipage}[c]{1\textwidth}
\begin{alltt}
\textcolor{blue}{\# Time scheme explicit or implicit \#}

{\bf{Scheme\_euler\_explicit}} \textit{my\_scheme}

{\bf{Read}} \textit{my\_scheme}

\{

\hspace{1cm}        \textcolor{blue}{\# Initial time \#}

\hspace{1cm}        \textcolor{blue}{\# Time step \#}

\hspace{1cm}        \textcolor{blue}{\# Output criteria \#}

\hspace{1cm}        \textcolor{blue}{\# Stop Criteria \#}

\}
\end{alltt}
\end{minipage}}
\vspace{0.1cm}



\fbox{ \begin{minipage}[c]{1\textwidth}
\begin{alltt}
\textcolor{blue}{\# Physical characteristics of medium \#}

{\bf{Fluide\_Incompressible}} \textit{my\_medium}

{\bf{Read}} \textit{my\_medium}

\{

\hspace{1cm}        ...

\}
\end{alltt}
\end{minipage}}
\vspace{0.1cm}



\fbox{ \begin{minipage}[c]{1\textwidth}
\begin{alltt}
\textcolor{blue}{\# Gravity vector definition \#}

{\bf{Uniform\_field}} \textit{my\_gravity}

{\bf{Read}} \textit{my\_gravity 2 0 -9.81}

\end{alltt}
\end{minipage}}
\vspace{0.1cm}



\fbox{ \begin{minipage}[c]{1\textwidth}
\begin{alltt}

\textcolor{blue}{\# Association between the different objects \#}

{\bf{Associate}} \textit{my\_problem my\_domain}

{\bf{Associate}} \textit{my\_problem my\_scheme}

{\bf{Associate}} \textit{my\_problem my\_medium}

{\bf{Associate}} \textit{my\_medium my\_gravity}
\end{alltt}
\end{minipage}}
\vspace{0.1cm}



\fbox{ \begin{minipage}[c]{1\textwidth}
\begin{alltt}
\textcolor{blue}{\# Discretization of the problem \#}

{\bf{Discretize}} \textit{my\_problem my\_discretization}
\end{alltt}
\end{minipage}}
\vspace{0.1cm}



\fbox{ \begin{minipage}[c]{1\textwidth}
\begin{alltt}
\textcolor{blue}{\# New domains for post-treatment \#}

\textcolor{blue}{\# By default each boundary condition of the domain is already extrated }

\textcolor{blue}{\hspace{0.2cm} with such name "my\_dom"\_boundaries\_"my\_BC" \#}

{\bf{Domaine}} \textit{plane}

{\bf{extraire\_surface}} 

\{

\hspace{1cm}        {\bf{domaine}} \textit{plane}

\hspace{1cm}        {\bf{probleme}} \textit{my\_probleme}

\hspace{1cm}        {\bf{condition\_elements}} (x>0.5)

\hspace{1cm}        {\bf{condition\_faces (1)}} 

\}
\end{alltt}
\end{minipage}}
\vspace{0.1cm}



\fbox{ \begin{minipage}[c]{1\textwidth}
\begin{alltt}
\textcolor{blue}{\# Problem description \#}

{\bf{Read}} \textit{my\_problem}

\{
\end{alltt}
\end{minipage}}
\vspace{0.1cm}



\fbox{ \begin{minipage}[c]{1\textwidth}
\begin{alltt}
\hspace{1cm}        \textcolor{blue}{\# hydraulic problem \#}

\hspace{1cm}        {\bf{Navier\_Stokes\_standard}}

\hspace{1cm}        \{

\hspace{2cm}                \textcolor{blue}{\# Choice of the pressure matrix solver \#}

\hspace{2cm}                {\bf{Solveur\_Pression}} \textit{solver} \{ ... \}

\hspace{2cm}                \textcolor{blue}{\# Diffusion operator \#}

\hspace{2cm}                {\bf{Diffusion}} \{ ... \}

\hspace{2cm}                \textcolor{blue}{\# Convection operator \#}

\hspace{2cm}                {\bf{Convection}} \{ ... \}

\hspace{2cm}                \textcolor{blue}{\# Sources \#}

\hspace{2cm}                {\bf{Sources}} \{ ... \}

\hspace{2cm}                \textcolor{blue}{\# Initial conditions \#}

\hspace{2cm}                {\bf{Initial\_conditions}} \{ ... \}

\hspace{2cm}                \textcolor{blue}{\# Boundary conditions \#}

\hspace{2cm}                {\bf{Boundary\_conditions}} \{ ... \}

\hspace{1cm}        \}
    \end{alltt}
\end{minipage}}
\vspace{0.1cm}



\fbox{ \begin{minipage}[c]{1\textwidth}
\begin{alltt}
\hspace{1cm}        \textcolor{blue}{\# Post\_processing description \#}

\hspace{1cm}        \textcolor{blue}{/* To know domains that can be treated directly, search in .err }

\hspace{1.6cm}       \textcolor{blue}{    output file: "Creating a surface domain named" */}

\hspace{1cm}        \textcolor{blue}{/* To know fields  that can be treated directly, search in .err }

\hspace{1.6cm}      \textcolor{blue}{     output file: "Reading of fields to be postprocessed" */}

\hspace{1cm}        {\bf{Post\_processing}}

\hspace{1cm}        \{

\hspace{2cm}                \textcolor{blue}{\# Definition of new fields \#}

\hspace{2cm}                {\bf{Definition\_Champs}} \{ ... \}


\hspace{2cm}                \textcolor{blue}{\# Probes \#}

\hspace{2cm}                {\bf{Probes}} \{ ... \}


\hspace{2cm}                \textcolor{blue}{\# Fields \#}

\hspace{2cm}                \textcolor{blue}{\# format default value: lml \#}

\hspace{2cm}                \textcolor{blue}{\# select 'lata' for VisIt tool or 'MED' for Salom\'e \#}

\hspace{2cm}                {\bf{format lata}}

\hspace{2cm}                {\bf{fields dt\_post}} 1. \{ ... \}


\hspace{2cm}                \textcolor{blue}{\# Statistical fields \#}

\hspace{2cm}                {\bf{Statistiques dt\_post}} 1. \{ ... \}

\hspace{1cm}        \} 
\end{alltt}
\end{minipage}}
\vspace{0.1cm}



\fbox{ \begin{minipage}[c]{1\textwidth}
\begin{alltt}
    \textcolor{blue}{\# Saving and restarting process \#}

    [{\bf{sauvegarde binaire}} \textit{datafile}{\bf{.sauv}}]

    [{\bf{resume\_last\_time binaire}} \textit{datafile}{\bf{.sauv}}]
\end{alltt}
\end{minipage}}
\vspace{0.1cm}


\fbox{ \begin{minipage}[c]{1\textwidth}
\begin{alltt}
\textcolor{blue}{\# End of the problem description block \#}

\}
\end{alltt}
\end{minipage}}
\vspace{0.1cm}



\fbox{ \begin{minipage}[c]{1\textwidth}
\begin{alltt}
\textcolor{blue}{\# The problem is solved with \#}

{\bf{Solve}} \textit{my\_problem}
\end{alltt}
\end{minipage}}
\vspace{0.1cm}



\fbox{ \begin{minipage}[c]{1\textwidth}
\begin{alltt}
\textcolor{blue}{\# Not necessary keyword to finish \#}

{\bf{End}}
\end{alltt}
\end{minipage}}


%%%%%%%%%%%%%%%%%%%%%%%%%%%%%%%%%%%%%%%%%%%%%%%%%%%%



%%%%%%%%%%%%%%%%%%%%%%%%%%%%%%%%%%%%%%%%%%%%%%%%%%%%
\subsection{Basic rules}
%%%%%%%%%%%%%%%%%%%%%%%%%%%%%%%%%%%%%%%%%%%%%%%%%%%%
There is no line concept in \trust.\\

Data files uses \underline{blocks}. They may be defined using the braces:\\
    \begin{center}
    \fbox{ \begin{minipage}[c]{0.5\textwidth}
    \begin{alltt}
    \{

    \hspace{1cm}    a block

    \}
    \end{alltt}
    \end{minipage}}
    \end{center}
\bigskip{}


%%%%%%%%%%%%%%%%%%%%%%%%%%%%%%%%%%%%%%%%%%%%%%%%%%%%



%%%%%%%%%%%%%%%%%%%%%%%%%%%%%%%%%%%%%%%%%%%%%%%%%%%%
\subsection{Objects notion}
%%%%%%%%%%%%%%%%%%%%%%%%%%%%%%%%%%%%%%%%%%%%%%%%%%%%
\textbf{Objects} are created in the data set as follows:
    \begin{center}
    \fbox{ \begin{minipage}[c]{0.5\textwidth}
    \begin{alltt}
        [ {\bf{export}} ] {\textit{ Type identificateur}}
    \end{alltt}
    \end{minipage}}
    \end{center}

\begin{itemize}
\item \textbf{export}: if this keyword is included, \textit{identificateur} (identifier) will have a global range, if not, its range will be applied to the block only (the associated object will be destroyed on exiting the block).
\item \textit{Type}: must be a type of object recognised by \trust, correspond to the C++ classes. The list of recognised types is given in the file hierarchie.dump.
\item \textit{identificateur}: the identifier of the object type \textit{Type} created, correspond to an instancy of the C++ class \textit{Type}. \trust exits in error if the identifier has already been used.
\end{itemize}

There are several object types. Physical objects, for example:

\begin{itemize}
%\item A block object (keyword \textbf{Pave}) is defined by its origin and dimensions (keyword \textbf{origine (origin)} and \textbf{longueurs (length)}). Discretization is given by the \textbf{nombre\_de\_noeuds (node number)} in each direction.
\item A \textbf{Fluide\_incompressible} (incompressible\_Fluid) object. This type of object is defined by its physical characteristics (its dynamic viscosity $\mu$ (keyword \textbf{mu}), its density $\rho$ (keyword \textbf{rho}), etc...),
\item A \textbf{Domaine}.
\end{itemize}

More abstract object types also exist:

\begin{itemize}
\item A \textbf{VDF} or \textbf{VEF} according to the discretization type,
\item A \textbf{Scheme\_euler\_explicit} to indicate  the scheme type,
\item A \textbf{Solveur\_pression} to denote the pressure system solver type,
\item A \textbf{Champ\_Uniforme} to define, for example, the gravity field.
\end{itemize}
%%%%%%%%%%%%%%%%%%%%%%%%%%%%%%%%%%%%%%%%%%%%%%%%%%%%



%%%%%%%%%%%%%%%%%%%%%%%%%%%%%%%%%%%%%%%%%%%%%%%%%%%%
\subsection{Interpretors notion}
%%%%%%%%%%%%%%%%%%%%%%%%%%%%%%%%%%%%%%%%%%%%%%%%%%%%
\textbf{Interprete } (interpretor) type objects are then used to handle the created objects with the following syntax:
    \begin{center}
    \fbox{ \begin{minipage}[c]{0.5\textwidth}
    \begin{alltt}
        {\bf{\textit{Type\_interprete}}} \textit{argument}
    \end{alltt}
    \end{minipage}}
    \end{center}

\begin{itemize}
\item \textit{Type\_interprete}: any type derived from the \textbf{Interprete} (Interpretor) type recognised by \trust. In this manual, they are written in \textbf{bold}. You can enlight them in your file editor with the command (details in section \ref{Run}):\\
\texttt{> trust -config nedit|vim|emacs}
\item \textit{argument}: an argument may comprise one or several object identifiers and/or one or several data blocks.
\end{itemize}

Interpretors allow some operations to be carried out on objects.\\

Currently available general interpretors include \textbf{Read}, \textbf{Read\_file}, \textbf{Ecrire} (Write), \textbf{Ecrire\_fichier} (Write\_file), \textbf{Associate}.
%%%%%%%%%%%%%%%%%%%%%%%%%%%%%%%%%%%%%%%%%%%%%%%%%%%%



%%%%%%%%%%%%%%%%%%%%%%%%%%%%%%%%%%%%%%%%%%%%%%%%%%%%
\subsection{Example}
%%%%%%%%%%%%%%%%%%%%%%%%%%%%%%%%%%%%%%%%%%%%%%%%%%%%
%\begin{itemize}
%\item A data set to write Ok on screen:
A data set to write Ok on screen:
    \begin{center}
    \fbox{ \begin{minipage}[c]{0.9\textwidth}
    \begin{alltt}
    {\bf{Nom}} a\_name    \hspace{0.5cm}       \textcolor{blue}{\# Creation of an object type. Name identifier a\_name \#}

    {\bf{Read}} a\_name Ok        \textcolor{blue}{\# Allocates the string "Ok" to a\_name \#}

    {\bf{Ecrire}} a\_name     \hspace{0.2cm}     \textcolor{blue}{\# Write a\_name on screen \#}
    \end{alltt}
    \end{minipage}}
    \end{center}

%%%\item An incorrect data set:
%%%    \begin{center}
%%%    \fbox{ \begin{minipage}[c]{0.5\textwidth}
%%%    \begin{alltt}
%%%    {\bf{Domaine}} truc

%%%    ...

%%%    {\bf{Probleme}} truc   \textcolor{blue}{\# TRUST exits in error \#}
%%%    \end{alltt}
%%%    \end{minipage}}
%%%    \end{center}

%%%A possible correction:
%%%    \begin{center}
%%%    \fbox{ \begin{minipage}[c]{0.9\textwidth}
%%%    \begin{alltt}
%%%    \{

%%%    {\bf{Domaine}} truc

%%%    \}   \hspace{2.6cm}   \textcolor{blue}{\# The domain truc is destroyed \#}

%%%    {\bf{Probleme}} truc  \textcolor{blue}{\# this is correct because truc is not used any more \#}
%%%    \end{alltt}
%%%    \end{minipage}}
%%%    \end{center}


%\item One data set nesting another:
%    \begin{center}
%    \fbox{ \begin{minipage}[c]{0.9\textwidth}
%    \begin{alltt}
%    {\bf{Read\_file}} fichier\_inclus ; 

%    \textcolor{blue}{\# you should use {\bf{export}} in the fichier\_inclus to export identifiers \#}
%    \end{alltt}
%    \end{minipage}}
%    \end{center}

%example of the fichier\_inclus file: 
%    \begin{center}
%    \fbox{ \begin{minipage}[c]{0.9\textwidth}
%    \begin{alltt}
%    {\bf{Dimension}} 2

%    {\bf{export Domaine}} dom

%    {\bf{export Probleme\_hydraulique}} pb
%    \end{alltt}
%    \end{minipage}}
%    \end{center}
%\end{itemize}
%%%%%%%%%%%%%%%%%%%%%%%%%%%%%%%%%%%%%%%%%%%%%%%%%%%%



%%%%%%%%%%%%%%%%%%%%%%%%%%%%%%%%%%%%%%%%%%%%%%%%%%%%
\subsection{Important remarks}
%%%%%%%%%%%%%%%%%%%%%%%%%%%%%%%%%%%%%%%%%%%%%%%%%%%%
\begin{enumerate}
\item To insert \underline{comments} in the data set, use \textcolor{blue}{\# .. \#} (or \textcolor{blue}{/* ... */}), the character \textcolor{blue}{\#} must always be enclosed by blanks.
\item The comma separates items in a list (a comma must be enclosed with spaces or a new line).
\item Interpretor keywords are recognised indiscriminately whether they are written in lower and/or upper case.
\item \textbf{On the contrary, object names (identifiers) are recognised differently if they are written in upper or lower case.}
\item \textbf{In the following description, items (keywords or values) enclosed by [ and ] are optional.}
\end{enumerate}
%%%%%%%%%%%%%%%%%%%%%%%%%%%%%%%%%%%%%%%%%%%%%%%%%%%%



%%%%%%%%%%%%%%%%%%%%%%%%%%%%%%%%%%%%%%%%%%%%%%%%%%%%%%%%%%%%%%%
\section{Running data file}\label{Run}
%%%%%%%%%%%%%%%%%%%%%%%%%%%%%%%%%%%%%%%%%%%%%%%%%%%%%%%%%%%%%%%
To use \trust, your shell must be "bash". So ensure you are in the right shell:
\begin{verbatim}
> echo $0
/bin/bash
\end{verbatim}

To run your data file, you must initialize the TRUST environment using the following command:
\begin{verbatim}
> source $my_path_to_TRUST_installation/env_TRUST.sh
source $my_path_to_TRUST_installation/env/env_TRUST.sh
TRUST vX.Y.Z support : triou@cea.fr
Loading personal configuration /$path_to_my_home_directory/.perso_TRUST.env
\end{verbatim}

Then you can run your calculation:
\begin{verbatim}
> cd $my_test_directory
> trust [-evol] my_data_file
\end{verbatim}

where "trust" command call the "trust" script.
You can have the list of his options with:
\begin{verbatim}
> trust -help
\end{verbatim}
or
\begin{verbatim}
> trust -h
\end{verbatim}

Here is a panel of available options:
\begin{verbatim}
Usage: trust [option] datafile [nb_cpus] [1>file.out] [2>file.err]
Where option may be:
-help|-h                 : List options.
-index                   : Access to TRUST ressources index.
-doc                     : Access to TRUST user's manual.
-config nedit|vim|emacs  : Configure nedit or vim or emacs with TRUST keywords.
-mesh                    : Visualize the mesh(es) contained in the data file.
-evol                    : Monitor the TRUST calculation (new IHM).
-wiz                     : Set a TRUST datafile from a MED file (new IHM).
-prm                     : Write a prm file and generate the corresponding pdf 
                           file in the build directory.
-clean                   : Clean the current directory from all the generated 
                           files by TRUST.
-copy                    : Copy the test case datafile from the TRUST database 
                           under the present directory. 
-check all|testcase|list : Check the non regression of all the test cases or a 
                           single test case or a list of tests cases specified 
                           in a file.
-check function|class|method::class : Check the non regression of a list of
                           tests cases covering a function, a class or a 
                           class method.
-gdb                     : Run under gdb debugger.
-valgrind                : Run under valgrind.
-valgrind_strict         : Run under valgrind with no suppressions. 
-create_sub_file         : Create a submission file only. 
-prod                    : Create a submission file and submit the job on the 
                           main production class with exclusive resource. 
-help_trust              : Print options of 
                           TRUST_EXECUTABLE [CASE[.data]] [options]. 
\end{verbatim}

%-monitor                : Run and monitor the progress of the TRUST calculation.
%-probes                 : Monitor the TRUST calculation only.
%-bigmem                 : Create a submission file and submit the job on the big 
%                          memory production class. 
%-queue queue            : Create a submission file with the specified queue and 
%                          submit the job. 
%-c ncpus                : Use ncpus CPUs allocated per task for a parallel 
%                          calculation. 

\vspace{0.5cm}
To learn how to use the "\textbf{-evol}" option, you can go to see the first exercise of the tutorial: \href{TRUST_tutorial.pdf\#exo1}{Obstacle}.

