Dans le cas particulier trait�, on consid�re une masse volumique
constante $\rho=2$, un angle $\alpha=30^\circ$ ($\sin\alpha=1/2$ et
$\cos\alpha=\sqrt{3}/2$) des forces de frottements parall�le et
perpendiculaire $F_1=1$ et $F_2=2$ et un champ de vitesse tel que
$U_0=1$.

La force de frottement correspondante est
\begin{equation}
- \boldsymbol{\Lambda} \cdot \rho \boldsymbol{u} = \frac{\sqrt{3}}{4} \, \boldsymbol{e}_x - \frac{7}{4} \, \boldsymbol{e}_y
\end{equation}

Le champ de pression correspondant est le suivant~:
\begin{equation}
P(x,y) = \frac{\sqrt{3}}{4} \, x - \frac{7}{4} \, y + cte
\end{equation}

