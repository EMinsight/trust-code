Dans cette partie, on consid�re le cas o� le coefficient de frottement
$\lambda$ est inversement proportionnel au rayon :
\begin{equation}
\lambda(r) = \frac{\lambda_1}{r}
\end{equation}

Dans ce cas, la solution de l'�quation diff�rentielle
(\ref{eq:uThetaDiff}) est la suivante :
\begin{equation}
u_\theta(r) = U_1 \, \sqrt{r}
\end{equation}
o� $U_1$ est une constante d�pendant du d�bit.

\medskip

La force de frottement et la perte de pression correspondantes sont
donn�es par
\begin{equation}
- \lambda \, (u_\theta)^2 \, \boldsymbol{e}_\theta = - \frac{\lambda_1 \, {U_1}^2}{r} \, \boldsymbol{e}_\theta 
\end{equation}
\begin{equation}
\Delta P_1 = \lambda_1 \, {U_1}^2 \, \frac{\pi}{2}
\end{equation}

\bigskip

Dans la suite, on trace les champs de pression et de vitesse calcul�s
par GENEPI.  Le champ de vitesse analytique est impos� sur les parois
int�rieure et ext�rieure et en entr�e alors que la presion est impos�e
nulle en sortie.
