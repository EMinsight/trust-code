\section{Syntax to define a mathematical function\label{parser}}
In a mathematical function,
used for example in field definition, it's possible to use the predifined function (an object parser is used to
evaluate the functions) :

ABS\ \ \ \ : absolute value function

COS \ \ \ \ : cosinus function

SIN\ \ \ \ : sinus function

TAN\ \ \ \ : tan function

ATAN\ \ : arctan function

EXP\ \ \ \ : exponential function

LN\ \ \ \ : neperian logaithm function

SQRT \ \ : root mean square function

INT\ \ \ \ : integer function

ERF\ \ \ \ : erf function

RND(x)\ \ : random function (values between 0 and x)

COSH\ \ \ \ : hyperbolic cosinus function

SINH\ \ \ \ : hyperbolic sinus function

TANH\ \ \ \ : hyperbolic tangent function

ACOS\ \ \ \ : inverse cosinus function

ATANH\ \ : inverse hyperbolic tangent function

NOT(x)\ \ : not equal to x 

x\_AND\_y \ \ : and function (returns 1 if x and y true else 0)

x\_OR\_y\ \ : or function (returns 1 if x or y true else 0)

x\_GT\_y\ \ : greater to (returns 1 if x{\textgreater}y else 0)

x\_GE\_y\ \ : greater or equal to (returns 1 if x{\textgreater}=y else 0)

x\_LT\_y\ \ : lesser to (returns 1 if x{\textless}y else 0)

x\_LE\_y\ \ : lesser or equal to (returns 1 if x{\textless}=y else 0)

x\_MIN\_y \ \ \ \ \ : minimum of x and y

x\_MAX\_y \ \ \ \ : maximum of x and y

x\_MOD\_y \ \ \ \ : modular division of x per y

x\_EQ\_y \ \ \ \ \ \ \ \ : equal to (returns 1 if x=y else 0)

x\_NEQ\_y \ \ \ \ \ : not equal to (returns 1 if x!=y else 0) 


\bigskip

You can also use the following operations:

+\ \ : addition

{}- \ \ : substraction

/ \ \ : division

*\ \ : multiplication

\%\ \ : modulo

\$\ \ : max

\^{} \ \ : power

{\textless}\ \ : lesser than

{\textgreater}\ \ : greater than

[\ \ : less or equal to

]\ \ : greater of equal to


\bigskip

You can also use the following constants:

Pi \ \ : pi value (3,1415{\dots})


\bigskip

The variables which can be used are:

x,y,z \ \ : coordinates 

t \ \ : time


\bigskip

{\bfseries
Examples:}

Champ\_front\_fonc\_txyz\index{Champ\_front\_fonc\_txyz} \ 2 \ cos(y+x\^{}2) \ t+ln(y)

Champ\_fonc\_xyz\index{xyz} dom 2 tanh(4*y)*(0.95+0.1*rnd(1)) 0.


\bigskip

{\bfseries
Possible error:}

Champ\_fonc\_txyz 1 \ cos(10*t)*(1{\textless}x{\textless}2)*(1{\textless}y{\textless}2)

Previous line is wrong. It should be written:

Champ\_fonc\_txyz 1 \ cos(10*t)*(1{\textless}x)*(x{\textless}2)*(1{\textless}y)*(y{\textless}2)
