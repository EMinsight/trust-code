%%%%%%%%%%%%%%%%%%%%%%%%%%%%%%%%%%%%%%%%%%%%%%%%%%%%%%%%%%%%%%%
\section{Output files} \label{post}
%%%%%%%%%%%%%%%%%%%%%%%%%%%%%%%%%%%%%%%%%%%%%%%%%%%%%%%%%%%%%%%
After running, you will find different files in your directory. Here is a short explaination of what you will find in each type of file depending on its extension.\\

%\subsubsection{Input files:}
%\begin{longtable}{|l|c|}
%\hline
%\textbf{File}                       & \textbf{Contents} \\ \hline \hline
%my\_data\_file\textbf{.data}      & Data file \\ \hline
%*\textbf{.geom}, *\textbf{.bin}, *\textbf{.unv}, *\textbf{.med}     & Meshing \\ \hline
%*\textbf{.geo}                      & Instructions file \\ \hline
%*\textbf{.ssz}                      & Sub zones \\ \hline
%DOMAIN\_NAME\textbf{\_000n.Zones}   & Sub domains \\ \hline
%\end{longtable}

%\subsubsection{Output files:}

Even if you don't post-process anything, you will have output files which are listed here:

\begin{longtable}{|l|c|}
\hline \textbf{File}                                    & \textbf{Contents} \\ \hline \hline \endhead
\hline\multicolumn{2}{|c|}{\textcolor{olive}{... continued on next page ...}}  \\ \hline \endfoot
\hline \endlastfoot
\textit{my\_data\_file}\textbf{.dt\_ev}                        & Time steps, facsec, equation residuals \\ \hline
\textit{my\_data\_file}\textbf{.stop}                          & Stop file ('0', '1' or 'Finished correctly') \\ \hline
\textit{my\_data\_file}\textbf{.log}                           & Journal logging  \\ \hline
\textit{my\_data\_file}\textbf{.TU}                            & CPU performances \\ \hline
\textit{my\_data\_file}\textbf{\_detail.TU}                    & Statistics of execution \\ \hline
%\textit{my\_data\_file}\textbf{.progress}                      & Percent of calculation advancement \\ \hline
\textit{my\_data\_file\_problem\_name}\textbf{.sauv} or \textbf{.xyz}      & Saving 2D/3D results for resume \\
or \textit{specified\_name}\textbf{.sauv} or \textbf{.xyz}                 & (binary files) \\ \hline
\end{longtable}

and the listing of boundary fluxes where:
\begin{itemize}
\item \textit{my\_data\_file}\textbf{\_Contrainte\_visqueuse.out} correspond to the friction drag exerted by the fluid,%: $\int_S (-\mu \nabla(u) \cdot \vec{n}) dS$ in Newtons (if SI units used),
\item \textit{my\_data\_file}\textbf{\_Convection\_qdm.out} contains the momentum flow rate,
%: $\int_S (\rho u^2 \cdot \vec{n}) dS$ in Newtons (if SI units used),
\item \textit{my\_data\_file}\textbf{\_Debit.out} is the volumetric flow rate,%: $\int_S (u \cdot \vec{n}) dS$ in [$m^2.s^{-1}$] (if SI units used),
\item \textit{my\_data\_file}\textbf{\_Force\_pression.out} correspond to the pressure drag exerted by the fluid.%: $\int_S (P \cdot \vec{n}) dS$ in Newtons (if SI units used).
\end{itemize}

If you add post-processings in your data files, you will find:

\begin{longtable}{|l|c|}
\hline \textbf{File}                                    & \textbf{Contents} \\ \hline \hline
\textit{my\_data\_file}\textbf{.sons}                          & 1D probes list \\ \hline
\textit{my\_data\_file\_probe\_name}\textbf{.son}              & 1D results with probes \\ \hline
\textit{my\_data\_file\_probe\_name}\textbf{.plan}             & 3D results with probes\\ \hline
\textit{my\_data\_file}\textbf{.lml} \textit{(default format)}             & \\
\textit{my\_data\_file}\textbf{.lata} \textit{(with all *.lata.* files)}   & \\
\textit{my\_data\_file}\textbf{.med}                                       & 2D/3D results \\
or \textit{specified\_name}\textbf{.lml} or \textbf{.lata} or \textbf{.med}  & \\ \hline
\end{longtable}

The sceen outputs are automatically redirected in \textit{my\_data\_file}\textbf{.out} and \textit{my\_data\_file}\textbf{.err} files if you run a parallel calculation or if you use the "\textbf{-evol}" option of the "trust" script.\\
Else you can redirect them with the following command:
\begin{verbatim}
# if not already done
> source $my_path_to_TRUST_installation/env_TRUST.sh
# then
> trust my_data_file 1>file_out.out 2>file_err.err
\end{verbatim}

In the .out file, you will find the listing of physical infos with mass balance and in the .err file, the listing of warnings, errors and domain infos.




%%%%%%%%%%%%%%%%%%%%%%%%%%%%%%%%%%%%%%%%%%%%%%%%%%%%%%%%%%%%%%%
\section{Tools}
%%%%%%%%%%%%%%%%%%%%%%%%%%%%%%%%%%%%%%%%%%%%%%%%%%%%%%%%%%%%%%%

To open your 3D results in \textbf{lata} format, you can use \href{https://wci.llnl.gov/simulation/computer-codes/visit}{VisIt} which is an open source software included in \trust package. For that you may "source" \trust environment and launch VisIt:
\begin{verbatim}
# if not already done
> source $my_path_to_TRUST_installation/env_TRUST.sh
# then
> visit -o my_data_file.lata &
\end{verbatim}
To learn how to use it, you can do the \trust tutorial exercise \href{TRUST_tutorial.pdf\#exo1}{Flow around an obstacle}. \\

To open your 3D results in \textbf{med} format, you can also use \href{https://wci.llnl.gov/simulation/computer-codes/visit}{VisIt} or \href{http://www.salome-platform.org}{Salom\'e} or \href{http://www.paraview.org}{Paraview}.\\

Here are some actions that you can perform when your simulation is finished:
\begin{itemize}
\item To visualize the positions of your probes in function of the 2D/3D mesh, you can open your .son files at the same time of the .lata file in VisIt.
\item If you need more probes, you can create them with VisIt (if you have post-processed the good fields) or with MEDCoupling.
\item You can use the option "\textbf{-evol}" of the trust script, like:
\begin{verbatim}
trust -evol my_data_file
\end{verbatim}
and access to the probes or open VisIt for 2D/3D visualizations via this tool.
\end{itemize}



