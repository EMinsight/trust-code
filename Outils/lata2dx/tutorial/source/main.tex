\documentclass[a4paper,10pt,dvips]{article}
% Pour pouvoir taper les accents dans le code source
\usepackage[T1]{fontenc}
\usepackage[latin1]{inputenc}
\usepackage{times}
% Extension postscript
%\usepackage[pdftex]{graphicx}
\usepackage[dvips]{graphicx}
\usepackage{color}
\usepackage{subfigure}
\usepackage{array}
\usepackage[hmargin={3cm,3cm},vmargin={1cm,1cm}]{geometry}

% Pour encadrer des formules (ex : $\boxed{truc=\theta}$)
% et caracteres supplementaires (gras)
\usepackage{amsmath}
% Fontes pour ensemble des reels
\usepackage{amsfonts}
% definition de Beqnarray
%\usepackage{fancybox}
% Francisation du document
%\usepackage{babel}

%usepackage[backref,pdftex,colorlinks = true,urlcolor = black,linkcolor = black,anchorcolor = black,citecolor = black]{hyperref}

% Macros diverses
\begin{document}

\newcommand{\opendx}[0]{\textit{OpenDX} }
\newcommand{\triou}[0]{\textit{Trio-U} }

\title{TRIO-U documentation: Tutorial for OpenDX}
\author{Benoit MATHIEU}
\maketitle

\section{Introduction}

\opendx is a powerful open source data visualization software.
The software and other useful ressources can be found on
\hbox{http://www.opendx.org}.

If the software is properly installed (let us assume that
it sits on \textsf{\$DXROOT/}), then you will find
the hypertext documentation in \textsf{\$DXROOT/dx/html/allguide.htm}.
We suggest that you first look at this documentation. There
is a quick overview and a tutorial for you to get familiar with
the concepts and interface of the software.

If you already worked with the~\textit{AVS} software, you will
see that the philosophy of these softwares are quite similar.

In the beginning, you will not need to know much about programming
in~\opendx. \triou comes with several programs for
basic visualisation of your numerical simulations. Once you are
familiar with the graphical interface and concepts, you will
probably want to enhance your visualisations. It will be time
for you to look closer at the programs. \opendx is very flexible
and allows many operations on data, including statistics and
offers many different ways to represent datasets. We encourage
you to explore the possibilities of the software, since a better
visualisation often decreases the time you need to analyse the
results.

\section{Installation}

This tutorial, the \textsf{lata\_dx\_filter} tool and the \opendx
programs are located in the
\textsf{vobs/Pre\_Post\_Trio\_U/Outils/lata2dx/} directory.

We assume that you have a proper \triou view installed,
which has support for \opendx (the \textsf{lata2dx} directory
should be present). We also assume that you have \opendx installed
and present in the \textsf{\$PATH} variable.
(it should run by simply typing \textsf{dx} in a terminal).

You will first need to compile the \textsf{lata\_dx\_filter} tool.
Within the \triou environment, open a terminal window, enter the
\textsf{lata2dx} directory and type make. This will compile a small
executable file and place it together with some script files in the
\textsf{\$TRIO\_U\_ROOT/exec} directory.

You are now ready to start some visualizations.

\section{Quick start}

We will select an example data file, modify it to output the data
in \textsf{lata} format, run the simulation and view the results
with \opendx.

First, start \triou. Create a new study and copy the datafile
\textsf{Cavite\_paroi\_defilante\_quadra.data} found in the
\textsf{Test\_reference/Cavite\_paroi\_defilante\_quadra} directory.
Run it \textit{as is} if you want to be sure that \triou is properly
working.

Then edit the datafile and  change the \textsf{Postraitement} entry:
\vspace{1em}

{
\centering
\begin{tabular}{c|c}
original entry & new entry \\
\hline\\ 
\begin{minipage}{6cm}
\begin{verbatim}
Champs dt_post 0.1
{
  pression elem
  pression som
  vitesse elem
  vitesse som
  vorticite elem
}
\end{verbatim}
\end{minipage}
&
\begin{minipage}{6cm}
\begin{verbatim}
format lata
Champs dt_post 0.1
{
  pression elem
  pression som
  vitesse elem
  vitesse som
  vorticite elem
}
\end{verbatim}
\end{minipage}
\end{tabular}
\par
}
\vspace{1em}

You just have to add one line containing \textsf{format lata}. Then you
can run the simulation. The simulation should complete within a few minutes.

We will now launch \opendx to view the data. Open a terminal window in your
study and type the following command line: \textsf{lance\_dx~simple\_importer~\&}.
\opendx displays an image window and a control panel which has been designed
for basic \triou usage.

Whenever a problem happens, you will see a message window popup
describing the error and the program window will show you where the program
crashed. If this happens, you probably made a mistake in manipulating
\opendx and you won't need to look into the code. Close these two
windows, select \textsf{reset} in the \textsf{connection} menu and
start again.

Click on the right of the \textsf{Import lata file} box (the three dots).
In the filter box, change \textsf{*.dx} to \textsf{*.lata} and press enter.
You should now see the lata file (\textsf{your\_study.lata})in the selection box.
Select it and click ok. Make sure that the \textsf{quad} box is checked
and the \textsf{binary} box is unchecked in the control panel.
Now, open the \textsf{execute} menu in the image window and select 
\textsf{execute on change}. The program will run and the display will be
updated every time you change a display parameter. Now we need to adjust
the scaling for the computational domain to fit in the window.
In the \textsf{options} menu, select \textsf{reset}.
We can see the boundary of the computational domain. To see the mesh,
select \textsf{show connections} in the control panel.

In this basic control panel, you will find two types of
representations for field data (velocity, temperature and so on...).
One is a solid color field, and the other is a glyph (spheres of
varying size for scalar data and vectors for vector data).  Tick the
\textsf{Show color field} box. This will update the \textsf{Selection}
box and select a default field for display (eg. \textsf{positions},
which is of no use for us). Scroll in the selection box and draw the
\textsf{PRESSION\_ELEM} field, which represents the pressure at the
center of the elements. As you can see, \opendx shows a color field
which is constant on each mesh element. If you select
\textsf{PRESSION\_SOM} instead (which represents the pressure at the
mesh nodes), \opendx interpolates the data and shows a bilinear field
on each element.

The colorbar on the right shows the amplitude and color correspondance.
\opendx automaticaly adjusts the colormap to match the minimum and maximum
value of the field.

The other representation is called glyph in \opendx. Tick the
corresponding box (always do this first for the selection box to show
the correct values), and choose \textsf{PRESSION\_ELEM} in the
selection box. \opendx draws a sphere in the center of each element,
the diameter being a linear function of the pressure. If you select
\textsf{PRESSION\_SOM}, the sphere are drawn on the mesh nodes. Now,
select \textsf{VITESSE\_SOM}. You see the velocity field! The scaling
is also automaticaly
 chosen to fit the currently displayed data. Fixing
the scaling manually requires a more complete panel or some
programming from you.

We displayed the first timestep (just one step after initialisation).
To explore the time evolution, select \textsf{sequencer} in the \textsf{execute}
menu. This small VCR like panel allows you to navigate through the timesteps
(it's time to look at the online documentation, chapter~2, tutorial~I,
using the sequencer). If you reach the end of the simulation, hit the
stop button and select the correct field again...

\section{Visualizing front tracking results}

The first step consists in setting up the \triou to output the data in
\textsf{lata} format. Here is for example the postprocessing part of a data file:

\begin{verbatim}
        Postraitement
        {
                format lata
                Champs binaire dt_post 5.e-4
                {
                        masse_volumique elem
                        indicatrice elem
                        pression elem
                        vitesse som
                        temperature elem
                }
                Interfaces dt_post 5.e-4
                {
                        couleur
                        vitesse
                        force_tension
                        debit_massique
                        temperature
                }
        }
\end{verbatim}

The \textsf{binary} keyword in \textsf{champs} also holds for the interfaces.
Available keyword for the interfaces are listed above. If no keyword is given,
only the interfaces nodes and connectivity is output.

Once the simulation has been run, it's time to visualize the results.
There is a dedicated program to visualize front tracking simulations.
The program will be launched with \textsf{lance\_dx front\_tracking\_complet}.
As usual, be sure to check the element type and binary mode before executing,
or a (recoverable) error will occur.

The panel has the usual color field and glyph sections to visualize the
traditional fields. Color field and glyph scaling can be imposed to
have the same amplitude for all timesteps. For the color field, the minimum
and maximum value will be given. For the velocity field, the scaling gives
the length of a vector representing a velocity of~1~ms$^-1$ (if the domain
is~1~mm wide and the velocity is~0.1~mms$^-1$, then the vector will have a
length of 1/10th of the domain width).

Besides, there is a dedicated section for interfaces. A black line represents
the interfaces, and glyphs can be drawn to represent the interface nodes.
Velocity, surface tension and other data can also be represented with
glyphs. 

The last section allows you to save all images in the specified directory
(do not forget the trailing~\textbf{/}). Image resolution can be chosen
among standard values. Then, every time the program executes to display
an image, a copy is written to disk. The image filename is built using the
timestep number. 

\section{Important notices}

\opendx uses a data cache to conserve recently displayed data. This has two
consequences. First, the default setting of using~80\% of the memory can
lead to some trouble even for small data sets. The laucher that we provide
with \triou (\textsf{lance\_dx}) therefore sets the upper limit of memory usage to~160~MB for
\opendx. In some cases you will need more memory. If \opendx complains,
you will have to do the following:
\begin{itemize}
\item{\textsf{connection->disconnect from server, yes}}
\item{\textsf{connection->start server, options}}
\item{Type in the memory you want \opendx to allocate, in megabytes}
\item{\textsf{ok, connect}}
\end{itemize}
The other consequence is that if you visualize data while the simulation is still
running and you reach the last timestep, \opendx will refuse to go to the
next timestep, even if it has been saved to disk by \triou. \opendx remembers in
the fact that loading this timestep failed, and will not try again.
You will have to clear the cache (\textsf{connection->reset server}), and restart
execution (\textsf{execute->execute on change}).

The same applies if you restart a simulation. You will have to cleanup the directory
before to remove all \textsf{lata} files (we provide the script \textsf{clean\_trio\_data}
which will remove all files in the current directory but \textsf{.data}, 
\textsf{.net}, \textsf{.cfg}, \textsf{.mesh}, \textsf{.prep} and \textsf{.geom}. Be
careful, think twice before running this script). Then don't forget to reset the server
or you might see the results of the previous simulation. The more memory you
give to \opendx, the more data it will keep in the cache. This can be handy
because \opendx will be able to redisplay very quickly images that were recently
computed.

For the optimized \opendx programs to work properly, all the postprocessing data
must be output at the same time (same \textsf{dt\_post} for fields and interfaces).

If you get bored of the program window and message window popping up every time
you reach the end of the simulation, you can disable this in the options menu
of the message windows (just uncheck all options). Remember to check these
options again to solve problems.

The front tracking visualization program won't work if you didn't add the
\textsf{Interfaces} section to the postprocessing options of the \triou
datafile.

\subsection{Other programs}

\triou also comes with other programs dedicated to some specific visualizations.
Most of them will have to be adapted to fit you needs.

\textsf{lance\_dx front\_tracking\_couple} will launch a program suitable
for vizualisation of coupled front-tracking problems with a fluid,
interfaces and a wall having temperature. For this program to work
``as is'', the domain names in the \triou data file must be
\textsf{dom} and \textsf{dom\_solide} respectively for the fluid and
the wall.

\textsf{front\_tracking\_3d} allows easy 3d visualizations. It allows to
cut the domain with up to three planes along each direction and to
move the cut planes. Also, interfaces are represented with their
nodes, a wire mesh or a triangular shaded patch with optional
transparency.

Front\_tracking networks can automatically export rendered images.
Choose a destination directory in the control panel, tick the
\textsf{save images} box and choose an image resolution.  Every
rendered image will be saved with the following printf style
format:~\textsf{''img\%05d.tiff''}, where~\%d is the timestep number
(first timestep is number~0). After having activated this feature,
doing \textsf{options->reset} in the image window will resize the
window to the selected image resolution for you to check the aspect
ratio.

\textsf{front\_tracking\_multi} illustrates how to display different simulations
simultaneously on the same image (to compare simulation schemes for example).
\textsf{dtpost} must be constant, otherwise the displayed time won't match.

\textsf{front\_tracking\_stream} illustrates how to draw streamlines.
Origins of the streamlines are manually placed (\textsf{image
  window->options->view conttrol}, mode=cursors, then double click on
the image to place or remove a point).

\textsf{front\_tracking\_glyph} illustrates how to display figures on
the interface nodes for debugging purposes.

Feel free to modify the networks. 

\section{Implementation details}

The front tracking method is expected to be used heavily and on large
datasets. The visualization tools are therefore optimized for speed.
The main optimization consists in only loading the fields necessary
for the currently rendered image.  The fixed mesh geometry and the
first timestep with all components are imported once only and stay in
\opendx's cache. Then we take advandage of one feature of the
lata\_dx\_filter tool, which allows to import some specified field data
only, and not the geometry. The newly read field data is merged with
the previously read geometry with the \textsf{replace} tools in the network.
This allows a speedup of a factor~5 or more (yes, it will increase your
productivity!).

FAQ: sauvegarde
\section{Frequently asked questions}
\subsubsection{How do I save images to files}
\textsf{Image window>File>Save image}. Select image format, file name
and output resolution (tick \textsf{allow rerendering}).


\end{document}
