\begin{alltt}
#  
 Cas test Front-tracking discontinu VDF.

  Cas test avec interface liquide-vapeur "interf"
                solide mobile            "body"
                concentration
  Interface liquide-vapeur initiale : un demi-plan + une goutte
  Remaillage, barycentrage, lissage, test collision, gravite, 
  tension superficielle.
  Ecriture des resultats au format lata: un fichier lata avec
  les champs volumiques et les interfaces liquide-vapeur(lata1),
  un fichier avec uniquement le solide mobile (lata2)
 Les algorithmes de remaillage avec changement de connectivite
 ne sont pas strictement equivalents entre sequentiel et parallele.
 Il y a donc des ecarts entre le sequentiel et le parallele.
 PARALLEL RUNS
#
dimension 3
# Generic problem used for Front Tracking calculation # 
Probleme_FT_Disc_gen pb
Domaine DOM
# BEGIN MESH #
Mailler DOM
\{
    Pave pave1
    \{
        origine 0. 0. 0.
        {\bf{longueurs 0.04 0.04 0.12 }}
        nombre_de_noeuds 11 11 16
    \}
    \{
        {\bf{bord paroi X = 0.   0. <= Y <= 0.04 0. <= Z <= 0.12 }}
        {\bf{bord paroi X = 0.04 0. <= Y <= 0.04 0. <= Z <= 0.12 }}
        {\bf{bord paroi Y = 0.   0. <= X <= 0.04 0. <= Z <= 0.12 }}
        {\bf{bord paroi Y = 0.04 0. <= X <= 0.04 0. <= Z <= 0.12 }}
        {\bf{bord bas   Z = 0.   0. <= X <= 0.04 0. <= Y <= 0.04 }}
        {\bf{bord haut  Z = 0.12 0. <= X <= 0.04 0. <= Y <= 0.04 }}
    \}
\}
# END MESH #
# BEGIN PARTITION
Partition DOM
\{
    Partitionneur tranche \{ tranches 2 1 1 \}
    Larg_joint 2
    Nom_Zones DOM
\}
Fin
END PARTITION #
# BEGIN SCATTER
Scatter DOM.Zones DOM
END SCATTER #
VDF dis
Schema_Euler_explicite  sch
Lire sch
\{
    tinit 0.
    tmax  0.1
    dt_min 1.e-7
    dt_max 0.5e-2
    dt_impr 10.
    dt_sauv 100
    seuil_statio -1
\}
# First phase: liquid #
Fluide_Incompressible liquide
Lire liquide
\{
    mu  Champ_Uniforme 1 0.282e-3
    rho Champ_Uniforme 1 1000.
\}
# Second phase: gas #
Fluide_Incompressible gaz
Lire gaz
\{
    mu  Champ_Uniforme 1 0.282e-3
    rho Champ_Uniforme 1 100.
\}
# Definition of the two phase media #
Fluide_Diphasique fluide
Lire fluide
\{
    # Give a number for each phase #
    fluide0 liquide
    fluide1 gaz
    # Surface tension #
    sigma   Champ_Uniforme 1 0.05
\}
# Add a constituent #
Constituant constituant
Lire constituant \{ coefficient_diffusion Champ_Uniforme 1 1e-6 \}
# Gravity field #
Champ_Uniforme gravite
Lire gravite 3 0. 0. -9.81
Associate fluide gravite
# Navier Stokes equation #
Navier_Stokes_FT_Disc              hydraulique
# One equation for the two phase flow interface #
Transport_Interfaces_FT_Disc       interf
# One equation for a moving body #
Transport_Interfaces_FT_Disc       body
# One equation for the constituent #
Convection_Diffusion_Concentration concentration
Associate pb hydraulique
Associate pb interf
Associate pb body
Associate pb concentration
Associate pb DOM
Associate pb sch
Associate pb fluide
Associate pb constituant
Discretize pb dis
# Define the front tracking problem #
Lire pb
\{
    hydraulique
    \{
        # Turbulence model needed and zeroed for laminar flow #
        {\bf{# modele_turbulence nul #}}
        modele_turbulence sous_maille_wale
        \{
            Cw               1.e-16
            turbulence_paroi negligeable
        \}
        solveur_pression GCP \{ precond ssor \{ omega 1.5 \} seuil 1e-12 impr \}
        convection           \{ quick \}
        diffusion            \{ \}
        conditions_initiales \{ vitesse champ_uniforme 3 0. 0. 0. \}
        # Relation beetween Navier Stokes equation and interface equations #
        # The velocity field moves the gas-liquid interface #
        equation_interfaces_proprietes_fluide interf 
        # The body has an imposed velocity field, so moves the fluid #
        equation_interfaces_vitesse_imposee   body
        boundary_conditions
        \{
            # Outlet boundary condition for FT model #
            haut   Sortie_libre_rho_variable champ_front_uniforme 1 0.    
            paroi  paroi_fixe
            bas    Frontiere_ouverte_vitesse_imposee champ_front_uniforme 3 0.0 0.0 0.001
        \}
        Traitement_particulier \{ Ec \{ Ec periode 1.e-7 \} \}
    \}
    interf
    \{
        # Definition of the transport method of the interface: velocity #
        # from the Navier Stokes equation #
        methode_transport vitesse_interpolee hydraulique
        # Initial position of the water-gas interface and a drop of water #
        conditions_initiales 
        \{
            fonction z-0.03-((x-0.02)^2+(y-0.02)^2)*10 ,
            fonction ajout_phase0 (x-0.02)^2+(y-0.02)^2+(z-0.045)^2-(0.01)^2 {\bf{,}}
            {\bf{fonction ajout_phase0 (x-0.02)^2+(y-0.02)^2+(z-0.08)^2-(0.01)^2 }}
        \}
        # Options for the meshing algorithm #
        iterations_correction_volume 1
        n_iterations_distance 2
        remaillage 
        \{
            pas 0.000001
            nb_iter_remaillage 1
            critere_arete 0.35
            critere_remaillage 0.2
            pas_lissage 0.000001
            lissage_courbure_iterations 3
            lissage_courbure_coeff -0.1
            nb_iter_barycentrage 3
            relax_barycentrage 1
            facteur_longueur_ideale 0.85
            nb_iter_correction_volume 3
            seuil_dvolume_residuel 1e-12
        \}
        # Algorithm for the collision algorithm between interfaces #
        collisions
        \{
            active
            juric_pour_tout
            type_remaillage Juric \{ source_isovaleur indicatrice \}
        \}
        # Boundary condition, variable contact angle is possible #
        boundary_conditions
        \{
            paroi Paroi_FT_disc symetrie
            haut  Paroi_FT_disc symetrie
            bas   Paroi_FT_disc symetrie
        \}
    \}
    body
    \{
        # Initial position of the moving body #
        conditions_initiales 
            \{ fonction -(((x-0.02))^2+((y-0.02)/0.6)^2+((z-0.02)/0.6)^2-(0.015^2)) \}
        remaillage 
        \{
            pas 1e8
            nb_iter_remaillage 5
            critere_arete 0.5
            critere_remaillage 0.2
            pas_lissage -1
            nb_iter_barycentrage 5
            relax_barycentrage 1
            facteur_longueur_ideale 1
        \}
        boundary_conditions
        \{
            haut  Paroi_FT_disc symetrie
            paroi Paroi_FT_disc symetrie
            bas   Paroi_FT_disc symetrie
        \}
        # 2 methods to move the body: velocity(x,y,z)=f(x,y,z) or velocity(x,y,z)=f(t) #
        methode_transport vitesse_imposee  -(y-0.02)*10  (x-0.02)*10  0.
    \}
    # Constituent equation #
    concentration
    \{
        diffusion \{ negligeable \}
        convection \{ quick \}
        conditions_initiales \{ concentration champ_fonc_xyz DOM 1 
                                    EXP(-((x-0.02)^2+(y-0.02)^2+(z-0.03)^2)/0.03^2) \}
        boundary_conditions 
        \{
            haut  frontiere_ouverte C_ext Champ_Front_Uniforme 1 0.
            paroi paroi
            bas   paroi
        \}
    \}
    Postraitement
    \{
        Definition_champs 
        \{
            Energie_cinetique_hydro Reduction_0D
            \{
                methode somme_ponderee source Transformation 
                \{
                    # Ec=sum[0.5*rho*vol*(u^2+v^2+w^2)dV] #
                    methode formule expression 1 0.5*rho*u2_plus_v2_plus_w2
                    sources \{
                        Transformation 
                        \{
                            methode produit_scalaire sources 
                            \{
                                Interpolation \{
                                    localisation elem
                                    source refChamp \{ Pb_champ pb vitesse \} \} ,
                                Interpolation \{
                                    localisation elem
                                    source refChamp \{ Pb_champ pb vitesse \} \}
                            \}
                            nom_source u2_plus_v2_plus_w2
                        \} ,
                        refChamp \{ Pb_champ pb masse_volumique nom_source rho \}
                    \}
                \}
            \}
        \}
        Sondes 
        \{
            vitesse vitesse periode 1.e-7 point 1 0.02 0.02 0.03
            pression pression periode 1.e-7 point 1 0.02 0.02 0.03
            indicatrice_interf indicatrice_interf periode 1.e-7 point 1 0.02 0.02 0.03
            energie_cinetique energie_cinetique_hydro periode 1.e-7 point 1 0.02 0.02 0.03
        \}
        {\bf{format lata}}
        {\bf{Champs dt_post 0.01 }}
        \{
            {\bf{indicatrice_interf elem}}
            {\bf{concentration elem}}
            masse_volumique
        \}
    \}
    liste_postraitements
    \{
        # Another keywords to post process FT results #
        Postraitement_ft_lata liquid_gas
        \{
            {\bf{dt_post 0.01 }}
            nom_fichier liquid_gas
            format binaire
            print
            champs sommets \{ vitesse \}
            champs elements
            \{
                distance_interface_elem_interf
                distance_interface_elem_body
                indicatrice_interf
                pression
                concentration
                vitesse
            \}
            # Post process the moving grid of the interface #
            interfaces interf \{
                champs sommets \{ courbure vitesse \}
                champs elements \{ pe \}
            \}
        \}
        Postraitement_ft_lata body
        \{
            {\bf{dt_post 0.01 }}
            nom_fichier body
            format binaire
            print
            interfaces body \{ champs sommets \{ courbure \} \}
        \}
    \}
\}
Solve pb
Fin
\end{alltt}
