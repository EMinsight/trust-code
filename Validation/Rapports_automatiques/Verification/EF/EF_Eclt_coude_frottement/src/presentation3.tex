Dans cette partie, on consid�re le cas o� le coefficient de frottement
$\lambda$ est inversement proportionnel au cube du rayon :
\begin{equation}
\lambda(r) = \frac{\lambda_3}{r^3}
\end{equation}

Dans ce cas, la solution de l'�quation diff�rentielle
(\ref{eq:uThetaDiff}) est la suivante :
\begin{equation}
u_\theta(r) = U_3 \, r
\end{equation}
o� $U_3$ est une constante d�pendant du d�bit.

\medskip

La force de frottement et la perte de pression correspondantes sont
donn�es par
\begin{equation}
- \lambda \, (u_\theta)^2 \, \boldsymbol{e}_\theta = - \frac{\lambda_3 \, {U_3}^2}{r} \, \boldsymbol{e}_\theta 
\end{equation}
\begin{equation}
\Delta P_3 = \lambda_3 \, {U_3}^2 \, \frac{\pi}{2}
\end{equation}

\medskip

Dans le cas o� la vitesse est telle que
$\boldsymbol{u}=u_\theta(r)\,\boldsymbol{e}_\theta$, on montre que la
force visqueuse est donn�e par
\begin{equation}
\nabla\cdot \left( \mu \, \nabla \boldsymbol{u} \right) = \mu \, \frac{d}{dr} \left[ \frac{1}{r} \, \frac{d}{dr} \left( r \, u_\theta(r) \right) \right] \boldsymbol{e}_\theta
\end{equation}
o� $\mu$ est la viscosit� dynamique suppos�e �tre constante.

Avec le profil $u_\theta(r)$ trouv� dans ce cas particulier, il est
facile de v�rifier que la force visqueuse correspondante est nulle.

\bigskip

Dans la suite, on trace les champs de pression et de vitesse calcul�s
par GENEPI.  Le champ de vitesse analytique est impos� sur les parois
int�rieure et ext�rieure et en entr�e alors que la presion est impos�e
nulle en sortie.

Contrairement aux deux premiers cas, le champ de vitesse tend vers un
�tat stationnaire, m�me en sortie.  La stabilisation du champ de
vitesse en sortie est du au terme visqueux.
