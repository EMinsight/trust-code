%%%%%%%%%%%%%%%%%%%%%%%%%%%%%%%%%%%%%%%%%%%%%%%%%%%%%%%%%%%%%%%
\section{Solve}
%%%%%%%%%%%%%%%%%%%%%%%%%%%%%%%%%%%%%%%%%%%%%%%%%%%%%%%%%%%%%%%
Now that you have finish to specify all your computation parameters, you may add the \href{../../Outils/TRIOXDATA/XTriou/doc.pdf\#solve}{\textbf{"Solve"}} keyword at the end of your data file, in order to resolve your problem.
You may also add the \textbf{"End"} keyword to specify the end of your data file.

    \begin{center}
    \fbox{ \begin{minipage}[c]{0.5\textwidth}
    \begin{alltt}
    {\bf{Solve}} \textit{my\_problem}

    [{\bf{End}}]
    \end{alltt}
    \end{minipage}}
    \end{center}

For more details, see the \href{../../Outils/TRIOXDATA/XTriou/doc.pdf\#solve}{\trust Reference Manual}. \\

You can see methods to run your data file in section \ref{Run}.



%%%%%%%%%%%%%%%%%%%%%%%%%%%%%%%%%%%%%%%%%%%%%%%%%%%%%%%%%%%%%%%
\section{Stop running calculation} \label{stopfile}
%%%%%%%%%%%%%%%%%%%%%%%%%%%%%%%%%%%%%%%%%%%%%%%%%%%%%%%%%%%%%%%
Your calculation will automatically stops if it has reached:
\begin{itemize}
\item the end of the calculation time,
\item the maximal allowed cpu time,
\item the maximal number of iterations or
\item the threshold of convergence.
\end{itemize}

You may use the \textit{my\_data\_file}\textbf{.stop} file, if you want to stop properly your running calculation (i.e. with all saves).\\

When the simulation is running, you can see the "\textbf{0}" value in that file.
To stop it, put a "\textbf{1}" instead of the "\textbf{0}" and at the next iteration the calculation will stop properly.
When you don't change any thing to that file, at the end of the calculation, you can see that it is writen "\textbf{Finished correctly}".



%%%%%%%%%%%%%%%%%%%%%%%%%%%%%%%%%%%%%%%%%%%%%%%%%%%%%%%%%%%%%%%
\section{Save}
%%%%%%%%%%%%%%%%%%%%%%%%%%%%%%%%%%%%%%%%%%%%%%%%%%%%%%%%%%%%%%%
Automaticaly, \trust make backups during the calculation. The unknowns (velocity, temperature,...) are saved in:
\begin{itemize}
\item one \textbf{.xyz} file, happening:
    \begin{itemize}
    \item at the end of the calculation,
    \item but, user may disable it with the specific keyword "\href{../../Outils/TRIOXDATA/XTriou/doc.pdf\#ecriturelecturespecial}{\textbf{EcritureLectureSpecial 0}}" added just before the \textbf{"Solve"} keyword.
    \end{itemize}


\item one, or several in case of parallel calculation, \textbf{.sauv} files, happening:
    \begin{itemize}
    \item at the start of the calculation,
    \item at the end of the calculation,
    \item each 23 hours of CPU, to change it, uses \small \textbf{"periode\_sauvegarde\_securite\_en\_heure"} \normalsize keyword (default value 23 hours),
    \item user may also specify a time physical period with \textbf{"dt\_sauv"} keyword,
    \item periodically using \textbf{"tcpumax"} keyword for which calculation stops after the specified time (default value $10^{30}$), use it for calculation on CCRT/TGCC and CINES clusters for example.
    \end{itemize}
\end{itemize}


By default, the name for the \textbf{.sauv} files is \textbf{"filename\_problemname.sauv"} for sequential calculation, \textbf{"filename\_problemname\_000n.sauv"} for parallel calculation (one per process).
The format of theses files is binary and the files are appended during successive saves.\\

You can change the behaviour using the following keywords just before the \textbf{solve} instruction:
\begin{center}
\fbox{ \begin{minipage}[c]{1\textwidth}
\begin{alltt}
{\bf{sauvegarde \hspace{0.5cm} binaire|xyz}} \hspace{0.5cm} \textit{filename}{\bf{.sauv}}|\textit{filename}{\bf{.xyz}}
\end{alltt}
\end{minipage}}
\end{center}

with
%\begin{itemize}
%\item \textbf{"formatte"}: the format of the file ASCII instead of binary (\textbf{"binaire"} keyword),
\textbf{"xyz"}: the \textbf{.xyz} file is written instead of the \textbf{.sauv} files.\\
%\end{itemize}

\Note that, you can use \textbf{"sauvegarde\_simple"} instead of \textbf{"sauvegarde"} where the .sauv or .xyz file is deleted before saves, in order to keep disk space:
\begin{center}
\fbox{ \begin{minipage}[c]{1\textwidth}
\begin{alltt}
{\bf{sauvegarde\_simple \hspace{0.5cm} binaire|xyz}} \hspace{0.5cm} \textit{filename}{\bf{.sauv}}|\textit{filename}{\bf{.xyz}}
\end{alltt}
\end{minipage}}
\end{center}

For more details, see the \href{../../Outils/TRIOXDATA/XTriou/doc.pdf\#Pbbase}{\trust Reference Manual}. \\




%%%%%%%%%%%%%%%%%%%%%%%%%%%%%%%%%%%%%%%%%%%%%%%%%%%%%%%%%%%%%%%
\section{Restart}
%%%%%%%%%%%%%%%%%%%%%%%%%%%%%%%%%%%%%%%%%%%%%%%%%%%%%%%%%%%%%%%

To restart your calculation, you may:
\begin{itemize}
\item change your initial time, the new initial time will be the real final calculation time of the previous calculation (cf .err file),
\item change your final calculation time to the new wanted value and
\item add the following block just before the \textbf{"Solve"} keyword:
    \begin{center}
    \fbox{ \begin{minipage}[c]{0.8\textwidth}
    \begin{alltt}
    {\bf{reprise \hspace{0.5cm} binaire|xyz}} \hspace{0.5cm} \textit{filename}{\bf{.sauv}}|\textit{filename}{\bf{.xyz}}
    \end{alltt}
    \end{minipage}}
    \end{center}
\end{itemize}
\vspace{0.5cm}

You can restart your calculation:
\begin{itemize}
\item from .sauv file(s) (one file per process): you can only restart the calculation with the \textbf{same number of equations} on \textbf{the same number of processes},
\item or from a .xyz file: here you can restart your calculation by \textbf{changing the number of equations solved} and/or with a \textbf{different number of processes}.
\end{itemize}

Instead of \textbf{"reprise"} keyword, you can use \textbf{"resume\_last\_time"} where \textbf{tinit} is automatically set to the last time of saved files (but you may change \textbf{tmax}):
    \begin{center}
    \fbox{ \begin{minipage}[c]{0.92\textwidth}
    \begin{alltt}
    {\bf{resume\_last\_time \hspace{0.5cm} binaire|xyz}} \hspace{0.5cm} \textit{filename}{\bf{.sauv}}|\textit{filename}{\bf{.xyz}}
    \end{alltt}
    \end{minipage}}
    \end{center}

For examples, see the \href{TRUST_tutorial.pdf\#save_restart}{\trust tutorial} and the \href{../../Outils/TRIOXDATA/XTriou/doc.pdf\#Pbbase}{\trust Reference Manual}.\\

\Note that you can run a calculation with initial condition read into a save file (.xyz or .sauv) from a previous calculation using \href{../../Outils/TRIOXDATA/XTriou/doc.pdf\#champfoncreprise}{\textbf{Champ\_Fonc\_reprise}} or read a into a MED file with \href{../../Outils/TRIOXDATA/XTriou/doc.pdf\#champfoncmed}{\textbf{Champ\_Fonc\_MED}}.






