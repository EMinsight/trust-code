\begin{longtable}[hcr]{|c|c|c|}
\hline \textbf{Physical values}                        & \textbf{Keyword for field\_name}          & \textbf{Unit} \\ \hline \endhead
\hline\multicolumn{3}{|c|}{\textcolor{olive}{... continued on next page ...}}  \\ \hline \endfoot
\hline \hline \endlastfoot


Velocity                                        & \textbf{Vitesse} or \textbf{Velocity}     & $m.s^{-1}$ \\ \hline
Velocity residual                               & \textbf{Vitesse\_residu}                  & $m.s^{-2}$ \\ \hline
Kinetic energy per elements                     &                                           & \\
($0.5 \rho ||u_i||^2$)                          & \textbf{Energie\_cinetique\_elem}         & $kg.m^{-1}.s^{-2}$ \\ \hline
Total kinetic energy                            &                                           & \\
$\displaystyle \left( \frac{\sum_{i=1}^{nb\_elem} 0.5 \rho ||u_i||^2 vol_i}{\sum_{i=1}^{nb\_elem} vol_i} \right)$                            & \textbf{Energie\_cinetique\_totale}       & $kg.m^{-1}.s^{-2}$ \\ \hline
Vorticity                                       & \textbf{Vorticite}                        & $s^{-1}$ \\ \hline
Pressure in incompressible flow                 &                                           & \\
($P/\rho+gz$)                                   & \textbf{Pression} \footnote{The post-processed pressure is the pressure divided by the fluid's density ($P/\rho+gz$) on incompressible laminar calculation. For turbulent, pressure is $P/\rho+gz+2/3*k$ cause the turbulent kinetic energy is in the pressure gradient.}
                                                                                            & $Pa.m^3.kg^{-1}$ \\
For Front Tracking probleme                     &                                           & or \\
($P+\rho gz$)                                   &                                           &  $Pa$ \\ \hline
Pressure in incompressible flow                 &                                           &   \\
(P+$\rho gz$)                                   & \textbf{Pression\_pa} or \textbf{Pressure}         & $Pa$ \\ \hline
Pressure in compressible flow                   & \textbf{Pression}                         & $Pa$ \\ \hline
Hydrostatic pressure $(\rho g z)$               & \textbf{Pression\_hydrostatique}          & $Pa$ \\ \hline
Totale pressure (when                           &                                           & \\
quasi compressible model                        &                                           & \\
is used)=Pth+P                                  & \textbf{Pression\_tot}                    & $Pa$ \\ \hline
Pressure gradient                               &                                           & \\
($\nabla(P/\rho+gz)$)                           & \textbf{Gradient\_pression}               & $m.s^{-2}$ \\ \hline
Velocity gradient                               & \textbf{gradient\_vitesse}                & $s^{-1}$ \\ \hline
Temperature                                     & \textbf{Temperature}                      & $^o$C or K \\ \hline
Temperature residual                            & \textbf{Temperature\_residu}              & $^o$C.$s^{-1}$ or K.$s^{-1}$ \\ \hline
Phase temperature of                            &                                           & \\
a two phases flow                               & \textbf{Temperature\_EquationName}        & $^o$C or K \\ \hline
Mass transfer rate                              &                                           & \\
between two phases                              & \textbf{Temperature\_mpoint}              & $kg.m^{-2}.s^{-1}$ \\ \hline
Temperature variance                            & \textbf{Variance\_Temperature}            & $K^2$ \\ \hline
Temperature dissipation rate                    & \textbf{Taux\_Dissipation\_Temperature}   & $K^2.s^{-1}$ \\ \hline
Temperature gradient                            & \textbf{Gradient\_temperature}            & $K.m^{-1}$ \\ \hline
Heat exchange coefficient                       & \textbf{H\_echange\_Tref} \footnote{Tref indicates the value of a reference temperature and must be specified by the user. For example, H\_echange\_293 is the keyword to use for Tref=293K.}            & $W.m^{-2}.K^{-1}$ \\ \hline
Turbulent heat flux                             & \textbf{Flux\_Chaleur\_Turbulente}        & $m.K.s^{-1}$ \\ \hline
Turbulent viscosity                             & \textbf{Viscosite\_turbulente}            & $m^2.s^{-1}$ \\ \hline
Turbulent dynamic viscosity                     &                                           & \\
(when quasi compressible                        & \textbf{Viscosite\_dynamique\_turbulente} & $kg.m.s^{-1}$ \\
 model is used)                                 &                                           & \\ \hline
Turbulent kinetic energy                        & \textbf{K}                                & $m^2.s^{-2}$ \\ \hline
Turbulent dissipation rate                      & \textbf{Eps}                              & $m^3.s^{-1}$ \\ \hline
Turbulent quantities                            &                                           & \\
K and Epsilon                                   & \textbf{K\_Eps}                           & ($m^2.s^{-2}$ ,$m^3.s^{-1}$ ) \\ \hline
Residuals of turbulent quantities               &                                           & \\
K and Epsilon residuals                         & \textbf{K\_Eps\_residu}                   & ($m^2.s^{-3}$ ,$m^3.s^{-2}$ ) \\ \hline
Constituent concentration                       & \textbf{Concentration}                    & \\ \hline
Constituent concentration residual              & \textbf{Concentration\_residu}            & \\ \hline
Component velocity along X                      & \textbf{VitesseX}                         & $m.s^{-1}$ \\ \hline
Component velocity along Y                      & \textbf{VitesseY}                         & $m.s^{-1}$ \\ \hline
Component velocity along Z                      & \textbf{VitesseZ}                         & $m.s^{-1}$ \\ \hline
Mass balance on each cell                       & \textbf{Divergence\_U}                    & $m^3.s^{-1}$  \\ \hline
Irradiancy                                      & \textbf{Irradiance}                       & $W.m^{-2}$ \\ \hline
Q-criteria                                      & \textbf{Critere\_Q}                       & $s^{-1}$ \\ \hline
Distance to the wall $Y^+=yU/\nu$               &                                           & \\ 
(only computed on                               & \textbf{Y\_plus}                          & dimensionless \\ 
boundaries of wall type)                        &                                           &  \\ \hline
Friction velocity                               & \textbf{U\_star}                          & $m.s^{-1}$ \\ \hline
Cell volumes                                    & \textbf{Volume\_maille}                   & $m^3$ \\ \hline
Chemical potential                              & \textbf{Potentiel\_Chimique\_Generalise}  & \\ \hline
Source term in non                              &                                           & \\
Galinean referential                            & \textbf{Acceleration\_terme\_source}      & $m.s^{-2}$ \\ \hline
Stability time steps                            & \textbf{Pas\_de\_temps}                   & S \\ \hline
Listing of boundary fluxes                      & \textbf{Flux\_bords}                      & cf each *.out file \\ \hline
Volumetric porosity                             & \textbf{Porosite\_volumique}              & dimensionless \\ \hline
Distance to the wall                            & \textbf{Distance\_Paroi} \footnote{distance\_paroi is a field which can be used only if the mixing length model (see 2.15.1.2) is used in the data file.}              & $m$\\ \hline
Volumic thermal power                           & \textbf{Puissance\_volumique}             & $W.m^{-3}$ \\ \hline
Local shear strain rate defined as              &                                           & \\
$\sqrt{(2SijSij)}$                              & \textbf{Taux\_cisaillement}               & $s^{-1}$ \\ \hline
Cell Courant number (VDF only)                  & \textbf{Courant\_maille}                  & dimensionless \\ \hline
Cell Reynolds number (VDF only)                 & \textbf{Reynolds\_maille}                 & dimensionless \\ \hline
Viscous force                                   & \textbf{viscous\_force}                   & $kg.m^{2}.s^{-1}$ \\ \hline
Pressure force                                  & \textbf{pressure\_force}                  & $kg.m^{2}.s^{-1}$ \\ \hline
Total force                                     & \textbf{total\_force}                     & $kg.m^{2}.s^{-1}$ \\ \hline
Viscous force along X                           & \textbf{viscous\_force\_x}                & $kg.m^{2}.s^{-1}$ \\ \hline
Viscous force along Y                           & \textbf{viscous\_force\_y}                & $kg.m^{2}.s^{-1}$ \\ \hline
Viscous force along Z                           & \textbf{viscous\_force\_z}                & $kg.m^{2}.s^{-1}$ \\ \hline
Pressure force along X                          & \textbf{pressure\_force\_x}               & $kg.m^{2}.s^{-1}$ \\ \hline
Pressure force along Y                          & \textbf{pressure\_force\_y}               & $kg.m^{2}.s^{-1}$ \\ \hline
Pressure force along Z                          & \textbf{pressure\_force\_z}               & $kg.m^{2}.s^{-1}$ \\ \hline
Total force along X                             & \textbf{total\_force\_x}                  & $kg.m^{2}.s^{-1}$ \\ \hline
Total force along Y                             & \textbf{total\_force\_y}                  & $kg.m^{2}.s^{-1}$ \\ \hline
Total force along Z                             & \textbf{total\_force\_z}                  & $kg.m^{2}.s^{-1}$ \\ \hline
\end{longtable}

