Dans cette partie, on consid�re le cas o� le coefficient de frottement
$\lambda$ est constant :
\begin{equation}
\lambda(r) = \lambda_0
\end{equation}

Dans ce cas, la solution de l'�quation diff�rentielle
(\ref{eq:uThetaDiff}) est la suivante :
\begin{equation}
u_\theta(r) = \frac{U_0}{\sqrt{r}}
\end{equation}
o� $U_0$ est une constante d�pendant du d�bit.

\medskip

La force de frottement et la perte de pression correspondantes sont
donn�es par
\begin{equation}
- \lambda \, (u_\theta)^2 \, \boldsymbol{e}_\theta = - \frac{\lambda_0 \, {U_0}^2}{r} \, \boldsymbol{e}_\theta 
\end{equation}
\begin{equation}
\Delta P_0 = \lambda_0 \, {U_0}^2 \, \frac{\pi}{2}
\end{equation}

\bigskip

Dans la suite, on trace les champs de pression et de vitesse calcul�s
par GENEPI.  Le champ de vitesse analytique est impos� sur les parois
int�rieure et ext�rieure et en entr�e alors que la presion est impos�e
nulle en sortie.
