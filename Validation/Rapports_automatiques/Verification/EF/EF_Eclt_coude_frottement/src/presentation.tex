On calcule l'�coulement bidimensionnel dans un coude avec un terme de
frottement volumique.  Trois cas sont consid�r�s.  Dans les trois cas,
la force volumique est suppos�e �tre quadratique en la vitesse.  Dans
le premier cas, le coefficient de frottement est constant, dans le
deuxi�me cas il est inversement proportionnel au rayon et dans le
troisi�me cas, il est inversement proportionnel au carr� du rayon.
Dans tous les cas, il existe une solution analytique simple pour les
champs de vitesse et de pression et telles que le champ de vitesse est
orthoradial (la composante radiale de la vitesse est nulle).  Il se
trouve que, pour tous les cas, les champs de vitesse diff�rent mais
les champs de pression et de force de frottement sont identiques.

\medskip

On consid�re le cas o� l'�coulement est r�gi par l'�quation de bilan
de quantit� de mouvement suivante :
\begin{equation}
-\nabla P - \lambda \, |u_\theta| \, u_\theta \, \boldsymbol{e}_\theta = 0 
\end{equation}
o� $P$ est la pression, $\boldsymbol{e}_\theta$ est le vecteur
unitaire orthoradial, $u_\theta$ est la composante orthoradiale de la
vitesse et $\lambda$ est le coefficient de frottement que l'on suppose
ne d�pendre que de la coordonn�e radiale $r$.

On cherche des solutions � cette �quation qui soient telles que la
vitesse radiale soit nulle : $u_r=0$.  L'�quation de continuit�
$\nabla\cdot\boldsymbol{u}=0$ implique alors que la composante
orthoradiale de la vitesse d�pend uniquement du rayon $r$ :
\begin{equation}
\boldsymbol{u} = u_\theta(r) \, \boldsymbol{e}_\theta
\end{equation}

On suppose en outre que l'�coulement est tel que $u_\theta>0$.

En projection sur les directions radiale et orthoradiale, l'�quation
de bilan de quantit� de mouvement donne :
\begin{equation}
\frac{\partial P}{\partial r} = 0
\label{eq:dPdr}
\end{equation}
\begin{equation}
\frac{1}{r} \, \frac{\partial P}{\partial \theta} = -\lambda (u_\theta)^2
\label{eq:dPdtheta}
\end{equation}

En d�rivant l'�quation (\ref{eq:dPdtheta}) par rapport � $r$ et en
tenant compte de l'�quation (\ref{eq:dPdr}), on en d�duit l'�quation
suivante :
\begin{equation}
2 \, \frac{du_\theta}{dr} + \left( \frac{1}{r} + \frac{1}{\lambda} \, \frac{d\lambda}{dr} \right) u_\theta = 0
\label{eq:uThetaDiff}
\end{equation}

Une fois le profil du coefficient de frottement $\lambda(r)$ donn�, il
suffit de r�soudre cette �quation diff�rentielle pour d�terminer le
profil de vitesse $u_\theta(r)$.

\medskip

Une fois le profil $u_\theta(r)$ d�termin�, le profil de pression
$P(\theta)$ se d�duit de l'�quation (\ref{eq:dPdtheta}) :
\begin{equation}
P(\theta) = - \left[ r \, \lambda \, (u_\theta)^2 \right] \theta
\label{eq:PTheta}
\end{equation}
o� l'on rappelle que le terme entre crochets est une constante.
