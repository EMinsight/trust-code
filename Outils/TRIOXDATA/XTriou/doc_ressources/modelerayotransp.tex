



Modele\_Rayonnement\_Milieu\_Transparent mod

Read mod \{

nom\_pb\_rayonnant

problem\_name

fichier\_fij

file\_name

fichier\_face\_rayo

file\_name

[fichier\_matrice | fichier\_matrice\_binaire file\_name]

\}



nom\_pb\_rayonnant problem\_name : problem\_name is the name of the radiating fluid problem



fichier\_fij file\_name : file\_name is the name of the file which contains the shape factor matrix between all the faces.



fichier\_face\_rayo file\_name : file\_name is the name of the file which contains the radiating faces characteristics (area, emission value ...)



fichier\_matrice|fichier\_matrice\_binaire file\_name : file\_name is the name of the ASCII (or binary) file which contains the inverted shape factor matrix. It is an optional keyword, if not defined, the inverted shape factor matrix will be calculated and written in a file.



The two first files can be generated by a preprocessor, they allow the radiating face characteristics to be entered (set of faces considered to be uniform with respect to radiation for emission value, flux, etc.) and the form factors for these various faces. These files have the following format:



File on radiating faces:

N M			-> N nombre de faces rayonnantes (=bords) et

			(N is the number of radiating faces (=edges) and

                      -> M nombre de faces rayonnantes a emissivitee non nulle

			M equals the number of non-zero emission radiating faces

Nom(i) S(i) E(i)	-> Nom du bord i, surface du bord i, valeur de	

 (Name of the edge i, surface area of the edge i)

-> l'emissivite (comprise entre 0 et 1) (emission value (between 0 an 1))

Exemple:

13 4

Gauche  50.0 0.0

Droit1  50.0 0.5

Bas     10.0 0.0

Haut    10.0 0.0

Arriere  5.0 0.0

Avant    5.0 0.0

Droit2  30.0 0.5

Bas1    40.0 0.0

Haut1   20.0 0.0

Avant1  20.0 0.0

Arriere1 20.0 0.0

Entree  20.0 0.5

Sortie  20.0 0.5



File on form factors:

N       -> Nombre de faces rayonnantes (Number of radiating faces)

Fij     -> Matrice des facteurs de formes avec i,j entre 1 et N (Matrix of form factors where i, j between 1 and N)

Example:

13              

1.00 0.00 0.00 0.00 0.00 0.00 0.00 0.00 0.00 0.00 0.00 0.00 0.00

0.00 0.00 0.00 0.00 0.00 0.00 0.24 0.20 0.10 0.10 0.10 0.10 0.16

0.00 0.00 1.00 0.00 0.00 0.00 0.00 0.00 0.00 0.00 0.00 0.00 0.00

0.00 0.00 0.00 1.00 0.00 0.00 0.00 0.00 0.00 0.00 0.00 0.00 0.00

0.00 0.00 0.00 0.00 1.00 0.00 0.00 0.00 0.00 0.00 0.00 0.00 0.00

0.00 0.00 0.00 0.00 0.00 1.00 0.00 0.00 0.00 0.00 0.00 0.00 0.00

0.00 0.40 0.00 0.00 0.00 0.00 0.00 0.20 0.10 0.10 0.10 0.10 0.00

0.00 0.25 0.00 0.00 0.00 0.00 0.15 0.00 0.15 0.10 0.10 0.15 0.10

0.00 0.25 0.00 0.00 0.00 0.00 0.15 0.30 0.00 0.10 0.10 0.00 0.10

0.00 0.25 0.00 0.00 0.00 0.00 0.15 0.20 0.10 0.00 0.10 0.10 0.10

0.00 0.25 0.00 0.00 0.00 0.00 0.15 0.20 0.10 0.10 0.00 0.10 0.10

0.00 0.25 0.00 0.00 0.00 0.00 0.15 0.30 0.00 0.10 0.10 0.00 0.10

0.00 0.40 0.00 0.00 0.00 0.00 0.00 0.20 0.10 0.10 0.10 0.10 0.00



Caution:

a) The radiation model's precision is decided by the user when he/she names the domain edges. In fact, a radiating face is recognised by the preprocessor as the set of domain edges faces bearing the same name. Thus, if the user subdivides the edge into two edges which are named differently, he/she thus creates two radiating faces instead of one.

b) The form factors are entered by the user, the preprocessor carries out no calculations other than checking preservation relationships on form factors.

c) The fluid is considered to be a transparent gas.
